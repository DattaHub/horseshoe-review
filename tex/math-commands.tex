%
% Math commands by Thomas Minka
%
% Revised by Jyotishka Datta & Brandon Willard
%
% TODO: should/could we put includes for dependencies in here?
%
\usepackage{suffix}

\numberwithin{equation}{section}
\newcommand{\var}{{\rm var}}
\newcommand{\Tr}{^{\rm T}}
\newcommand{\rmlog}{\rm log}
\newcommand{\vtrans}[2]{{#1}^{(#2)}}
\newcommand{\kron}{\otimes}
\newcommand{\schur}[2]{({#1} | {#2})}
\newcommand{\schurdet}[2]{\left| ({#1} | {#2}) \right|}
\newcommand{\had}{\circ}
\newcommand{\diag}{{\rm diag}}
\newcommand{\invdiag}{\diag^{-1}}
\newcommand{\rank}{{\rm rank}}
% careful: ``null'' is already a latex command
\newcommand{\nullsp}{{\rm null}}
\newcommand{\tr}{{\rm tr}}
\renewcommand{\vec}{{\rm vec}}
\newcommand{\vech}{{\rm vech}}
\renewcommand{\det}[1]{\left| #1 \right|}
\newcommand{\pdet}[1]{\left| #1 \right|_{+}}
\newcommand{\pinv}[1]{#1^{+}}
\newcommand{\erf}{{\rm erf}}
\newcommand{\hypergeom}[2]{{}_{#1}F_{#2}}

% boldface characters
\renewcommand{\a}{{\bf a}}
\renewcommand{\b}{{\bf b}}
\renewcommand{\c}{{\bf c}}
\renewcommand{\d}{{\rm d}}  % for derivatives
\newcommand{\e}{{\bf e}}
\newcommand{\f}{{\bf f}}
\newcommand{\g}{{\bf g}}
\newcommand{\h}{{\bf h}}
%\newcommand{\k}{{\bf k}}
% in Latex2e this must be renewcommand
\renewcommand{\k}{{\bf k}}
\newcommand{\m}{{\bf m}}
\newcommand{\n}{{\bf n}}
%\renewcommand{\o}{{\bf o}}
\newcommand{\p}{{\bf p}}
%\newcommand{\q}{{\bf q}}
\renewcommand{\r}{{\bf r}}
\newcommand{\s}{{\bf s}}
\renewcommand{\t}{{\bf t}}
\renewcommand{\u}{{\bf u}}
%\renewcommand{\v}{{\bf v}}
\newcommand{\w}{{\bf w}}
\newcommand{\x}{{\bf x}}
\newcommand{\y}{{\bf y}}
%\newcommand{\z}{{\bf z}}
\newcommand{\A}{{\bf A}}
\newcommand{\B}{{\bf B}}
%\newcommand{\C}{{\bf C}}
\newcommand{\D}{{\bf D}}
\newcommand{\E}{\mathbb E}
\newcommand{\F}{{\bf F}}
%\newcommand{\G}{{\bf G}}
\renewcommand{\H}{{\bf H}}
\newcommand{\I}{{\bf I}}
\newcommand{\J}{{\bf J}}
\newcommand{\K}{{\bf K}}
\renewcommand{\L}{{\bf L}}
\newcommand{\M}{{\bf M}}
\newcommand{\Nor}{{\cal N}}  % for normal density
%\newcommand{\N}{{\bf N}}
\renewcommand{\O}{{\bf O}}
\renewcommand{\P}{\mathbb P}
\newcommand{\Q}{{\bf Q}}
\newcommand{\R}{{\bf R}}
%\renewcommand{\S}{{\bf S}}
\newcommand{\T}{{\bf T}}
\newcommand{\U}{{\bf U}}
\newcommand{\V}{\mathbb V}
\newcommand{\W}{{\bf W}}
\newcommand{\X}{{\bf X}}
\newcommand{\Y}{{\bf Y}}
\newcommand{\Z}{{\bf Z}}

% this is for latex 2.09
% unfortunately, the result is slanted - use Latex2e instead
%\newcommand{\bfLambda}{\mbox{\boldmath$\Lambda$}}
% this is for Latex2e
\newcommand{\bfLambda}{\boldsymbol{\Lambda}}

% Yuan Qi's boldsymbol
\newcommand{\bsigma}{\boldsymbol{\sigma}}
\newcommand{\balpha}{\boldsymbol{\alpha}}
\newcommand{\bpsi}{\boldsymbol{\psi}}
\newcommand{\bphi}{\boldsymbol{\phi}}
\newcommand{\bbeta}{\boldsymbol{\beta}}
%\newcommand{\Beta}{\boldsymbol{\eta}}
\newcommand{\btau}{\boldsymbol{\tau}}
\newcommand{\bvarphi}{\boldsymbol{\varphi}}
\newcommand{\bzeta}{\boldsymbol{\zeta}}
\newcommand{\bnabla}{\boldsymbol{\nabla}}
\newcommand{\blambda}{\boldsymbol{\lambda}}
\newcommand{\bLambda}{\boldsymbol{\Lambda}}

\newcommand{\btheta}{\boldsymbol{\theta}}
\newcommand{\bpi}{\boldsymbol{\pi}}
\newcommand{\bPi}{\boldsymbol{\Pi}}
\newcommand{\bxi}{\boldsymbol{\xi}}
\newcommand{\bSigma}{\boldsymbol{\Sigma}}

\newcommand{\bgamma}{\boldsymbol{\gamma}}
\newcommand{\bGamma}{\boldsymbol{\Gamma}}

\newcommand{\bmu}{\boldsymbol{\mu}}
\newcommand{\bnu}{\boldsymbol{\nu}}
\newcommand{\bPsi}{\boldsymbol{\Psi}}
\newcommand{\bepsilon}{\boldsymbol{\epsilon}}
\newcommand{\bOmega}{\boldsymbol{\Omega}}

\newcommand{\1}{{\bf 1}}
\newcommand{\0}{{\bf 0}}

%\newcommand{\comment}[1]{}

\newcommand{\bs}{\backslash}
\newcommand{\ben}{\begin{enumerate}}
\newcommand{\een}{\end{enumerate}}
\newcommand{\beq}{\begin{equation}}
\newcommand{\eeq}{\end{equation}}
\newcommand{\bde}{\begin{description}}
\newcommand{\ede}{\end{description}}

 \newcommand{\notS}{{\backslash S}}
 \newcommand{\nots}{{\backslash s}}
 \newcommand{\noti}{{\backslash i}}
 \newcommand{\notj}{{\backslash j}}
 \newcommand{\nott}{\backslash t}
 \newcommand{\notone}{{\backslash 1}}
 \newcommand{\nottp}{\backslash t+1}
% \newcommand{\notz}{\backslash z}

\newcommand{\notk}{{^{\backslash k}}}
%\newcommand{\noti}{{^{\backslash i}}}
\newcommand{\notij}{{^{\backslash i,j}}}
\newcommand{\notg}{{^{\backslash g}}}
\newcommand{\wnoti}{{_{\w}^{\backslash i}}}
\newcommand{\wnotg}{{_{\w}^{\backslash g}}}
\newcommand{\vnotij}{{_{\v}^{\backslash i,j}}}
\newcommand{\vnotg}{{_{\v}^{\backslash g}}}
\newcommand{\half}{\frac{1}{2}}
\newcommand{\quart}{\frac{1}{4}}
\newcommand{\msgb}{m_{t \leftarrow t+1}}
\newcommand{\msgf}{m_{t \rightarrow t+1}}
\newcommand{\msgfp}{m_{t-1 \rightarrow t}}

\newcommand{\proj}[1]{{\rm proj}\negmedspace\left[#1\right]}
\newcommand{\argmin}{\operatornamewithlimits{argmin}}
\newcommand{\argmax}{\operatornamewithlimits{argmax}}

\newcommand{\dif}{\mathrm{d}}
\newcommand{\abs}[1]{\lvert#1\rvert}
\newcommand{\norm}[1]{\lVert#1\rVert}
\newcommand{\vectornorm}[1]{\left|\left|#1\right|\right|}

\newcommand{\rnorm}{\mathcal{N}}
\newcommand{\bx}{{\bf x}}
\newcommand{\ba}{{\bf a}}
\newcommand{\bb}{{\bf b}}
\newcommand{\bc}{{\bf c}}
\newcommand{\bd}{{\bf d}}
\newcommand{\bX}{{\bf X}}
\newcommand{\by}{{\bf y}}
\newcommand{\IG}{\mathcal{IG}}
\newcommand{\dd}[2]{\frac{\partial #1}{\partial #2}}
\newcommand{\lhat}[1][i]{\hat\lambda_{#1}^{-1(g)}}
\newcommand{\what}[1][j]{\hat\omega_{#1}^{-1(g)}}
\newcommand{\bone}{{\bf 1}}
\newcommand{\Li}{\hat\Lambda^{-1(g)}}
\newcommand{\Oi}{\hat\Omega^{-1(g)}}
\newcommand{\id}{\stackrel{\mathrm{ind}}{\sim}}
\newcommand{\iid}{\stackrel{\mathrm{iid}}{\sim}}
\newcommand{\iidp}{\stackrel{\mathrm{P}}{=}}
\newcommand{\iidd}{\stackrel{\mathrm{D}}{=}}
\newcommand{\defeq}{\operatorname{:=}}

% the last {} is a hack for double subscript errors
\newcommand{\estHsp}{\ensuremath{{\hat{\theta}}_{HS+}}{}}
\newcommand{\estHs}{\ensuremath{{\hat{\theta}}_{HS}}{}}
\newcommand{\estJs}{\ensuremath{{\hat{\btheta}}_{JS}}{}}
\newcommand{\MSE}{\mathrm{MSE}}
%
% Meijer-G additions
%
%%\DeclarePairedDelimiterX\MeijerM[3]{\lparen}{\rparen}%
%%{\begin{smallmatrix}#1 \\ #2\end{smallmatrix}\delimsize\vert\,#3}
%%
%%\newcommand\MeijerG[8][]{%
%%  G^{\,#2,#3}_{#4,#5}\MeijerM[#1]{#6}{#7}{#8}}
%%
%%\WithSuffix\newcommand\MeijerG*[7]{G^{\,#1,#2}_{#3,#4}\MeijerM*{#5}{#6}{#7}}
%%% end Meijer-G
%%
%%%
%%% Generalized Hypergeometric Function (pFq)
%%%
%%\DeclarePairedDelimiterX\pFqM[3]{\lparen}{\rparen}%
%%{\begin{smallmatrix}#1 \\ #2\end{smallmatrix}\delimsize\vert\,#3}
%%
%%\newcommand\pFq[6][]{%
%%  {}_{#2}F_{#3}\pFqM[#1]{#4}{#5}{#6}}
%%
%%\WithSuffix\newcommand\pFq*[5]{{}_{#1}F_{#2}\pFqM*{#3}{#4}{#5}}
%%% end pFq
%%
%\newtheorem{theorem}{THEOREM}
%\numberwithin{theorem}{section}
%\newtheorem{Proof}{PROOF}
%\newtheorem{Def}{DEFINITION}
%\numberwithin{Def}{section}
%\newtheorem{remark}{REMARK}
%\numberwithin{remark}{section}
%\newtheorem{Qes}{Question}
%\newtheorem{proposition}{PROPOSITION}
%\numberwithin{proposition}{section}
%\newtheorem{lemma}{LEMMA}
%\numberwithin{lemma}{section}
%\newtheorem{Cor}{COROLLARY}
%\numberwithin{Cor}{section}
%\newtheorem{Exa}{Example}
%\newtheorem{Eq}{Equation}
%\newtheorem{assn}{ASSUMPTION}
%%\newtheorem{result}[theorem]{Result}
%\newtheorem{result}[theorem]{RESULT}
%%


%\theoremstyle{slplain}

\newtheoremstyle{mystyle}%                % Name
  {0.2in}%                                     % Space above
  {}%                                     % Space below
  {\sl}%                                     % Body font
  {}%                                     % Indent amount
  {\bfseries}%                            % Theorem head font
  {.}%                                    % Punctuation after theorem head
  { }%                                    % Space after theorem head, ' ', or \newline
  {}%                                     % Theorem head spec (can be left empty, meaning `normal')

\theoremstyle{mystyle}

\newtheorem{theorem}{Theorem}
\newtheorem{acknowledgement}[theorem]{Acknowledgement}
%\newtheorem{algorithm}[theorem]{Algorithm}
\newtheorem{axiom}[theorem]{Axiom}
\newtheorem{case}[theorem]{Case}
\newtheorem{claim}[theorem]{Claim}
\newtheorem{conclusion}[theorem]{Conclusion}
\newtheorem{condition}[theorem]{Condition}
\newtheorem{conjecture}[theorem]{Conjecture}
\newtheorem{corollary}[theorem]{Corollary}
\newtheorem{criterion}[theorem]{Criterion}
\newtheorem{definition}[theorem]{Definition}
\newtheorem{example}[theorem]{Example}
\newtheorem{exercise}[theorem]{Exercise}
\newtheorem{lemma}[theorem]{Lemma}
\newtheorem{notation}[theorem]{Notation}
\newtheorem{problem}[theorem]{Problem}
\newtheorem{proposition}[theorem]{Proposition}
\newtheorem{remark}[theorem]{Remark}
\newtheorem{solution}[theorem]{Solution}
\newtheorem{summary}[theorem]{Summary}


%%\usepackage{booktabs,array}
%%\def\Midrule{\midrule[\heavyrulewidth]}
%%\newcount\rowc
%
%\makeatletter
%\def\ttabular{%
%\hbox\bgroup
%\let\\\cr
%\def\rulea{\ifnum\rowc=\@ne \hrule height 1.0pt \fi}
%\def\ruleb{
%\ifnum\rowc=1\hrule height 1.0pt  \else
%%\ifnum\rowc=3\hrule height 0.0pt%\heavyrulewidth 
%\ifnum\rowc= 3  \hrule height 0.5pt \else%\heavyrulewidth 
%\ifnum\rowc= 5  \hrule height 0.5pt \else%\heavyrulewidth 
%\ifnum\rowc= 7  \hrule height 0.5pt \else%\heavyrulewidth 
%\ifnum\rowc= 9  \hrule height 0.5pt \else%\heavyrulewidth 
%\ifnum\rowc= 11  \hrule height 0.5pt %\heavyrulewidth 
%  \else \hrule height 0pt%\lightrulewidth
%\fi\fi\fi\fi\fi\fi}
%\valign\bgroup
%\global\rowc\@ne
%\rulea
%\hbox to 7em{\strut \hfill##\hfill}%
%\ruleb
%&&%
%\global\advance\rowc\@ne
%\hbox to 7em{\strut\hfill##\hfill}%
%\ruleb
%\cr}
%\def\endttabular{%
%\crcr\egroup\egroup}

